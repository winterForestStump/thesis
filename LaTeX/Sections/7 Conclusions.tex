\section{Conclusions}
The aim of this thesis was to explore and develop a RAG system to enhance the reading and understanding of financial annual reports using open-source libraries and models. The power of artificial intelligence, especially in finance, has opened up new possibilities for advanced tools like LLMs and RAGs systems to simplify complicated tasks. Corporate annual reports, including SEC Form 10-K filings, provide crucial insights into a company's financial health and operational dynamics but are often lengthy and intricate. This thesis sought to mitigate these challenges by leveraging the capabilities of LLMs combined with information retrieval.

We constructed the RAG pipeline using open-source tools and models, emphasizing the accessibility and cost-effectiveness of this approach. Key components of the pipeline included ChromaDB, LangChain and HuggingFace, which were used to deploy a vector database with embedded vectors of the companies' reports, create a retriever for the most relevant content to user queries, and use LLM to generate responses. Our system was tested locally (running the model with LlamaCpp) on a Google Colab virtual machine, avoiding third-party API calls and ensuring data security.

A significant part of our research involved experimenting with various distance metrics to determine the best fit for database similarity searches. The main experiments with RAG system involved two distinct question datasets: Regular Questions and FinanceBench Questions. The results were manually evaluated based on the quality of retrieval, faithfulness of answers (grounded on the provided context), and their overall usefulness (correctness, completeness, and relevance) for the Regular Questions answers and based on the usefulness and correctness for the FinanceBench answers.

Our findings indicate that the RAG system can provide reasonable and useful answers, adding significant value to the baseline closed-book LLM's answers by incorporating additional knowledge into the responses. However, it is not yet reliable enough to completely replace human involvement in the question-answering process. Human oversight remains crucial to verify the accuracy of the results, particularly given the tendency of LLMs to generate plausible but factually incorrect information.

The RAG system demonstrated notable improvements in the usefulness of the answers compared to settings without RAG, highlighting the critical role of retrieval quality. High-quality context retrieval directly correlates with the quality of the generated responses, underscoring the importance of continued advancements in retrieval techniques and the overall architecture of RAG systems.

This thesis also highlighted the cost considerations associated with LLMs. While utilizing open-source models can mitigate some expenses, the performance gap between these and proprietary models often necessitates additional resources for manual curating the results, fine-tuning the RAG pipeline. Despite these challenges, the adoption of open-source LLMs remains an attractive option due to their flexibility and customization potential.

The thesis refrained from pursuing predictive tasks such as fraud detection or price prediction, focusing instead on the hypothesis that individuals should make these predictions based on reliable information extracted by the RAG system. By doing so, the research aimed to create and evaluate a tool that supports users within the finance reports, including private investors, professional financial experts, and SEC employees, each benefiting from automated and accurate information retrieval.

In conclusion, while the RAG system we developed shows promise in facilitating the understanding of financial annual reports, significant work remains to be done. Enhancements in retrieval methods are paramount, as the quality of retrieved context fundamentally defines the response quality. Future research should focus on improving retrieval techniques, adopting sophisticated RAG architectures, and exploring new methodologies to enhance LLM's responses.

The overarching goal of this project was to demonstrate the viability of a open-source RAG system that could run locally on a user's work computer. This approach democratizes access to advanced artificial intelligence technologies. Our results indicate that while the system adds substantial value, further refinements are necessary to achieve a level of reliability that could reduce human oversight. As the field of artificial intelligence continues to evolve, the integration of LLMs with real-time, verified data sources presents a promising way for making financial information more accessible and actionable. The success of this project lays a foundation for future work aimed at refining and expanding the capabilities of RAG systems in finance and beyond.



